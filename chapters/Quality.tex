\section{Quality}
% De lezer leest in dit hoofdstuk hoe je de kwaliteit van de (tussen)producten uit de vorige paragraaf
% waarborgt. Dit kun je bijvoorbeeld beschrijven door:
% -
% - Of en in welke mate je gebruikmaakt van extern advies (bijv. iemand buiten je projectgroep die
% je (tussen)product controleert). Geef aan hoe je de feedback verwerkt;
% - Hoe je omgaat met regelgeving, veiligheid, programma van eisen etc;

% Te beschrijven welke software, materialen en andere hulpbronnen je gebruikt en hoet dit
% bijdraagt aan de kwaliteit van jullie projectresultaat.

% Om de kwaliteit voor het project te waarborgen zal er gebruik gemaakt worden van de software STM32CubeIDE, waarmee de stm32-microcontroller geprogrammeerd kan worden. De stm32-chip is gebruikelijk in de industrie, vanwege zijn manier van programmeren, zoals het instellen van de GPIO en klokfrequenties. Bovendien zou het realtime simulatiepakket Caspoc gebruikt kunnen worden. Hiermee kan de werking en aansturing van de motordriver getest worden. Daarnaast wordt KiCad 7.0 gebruikt voor het ontwerpen van PCB's. Kicad is opensource en beschikt over de functies om een PCB te ontwerpen voor de motordriver. Het bestellen van de PCB wordt bij Eurocircuits gedaan, vanwege afspraken die school heeft gemaakt. \\
% Voor het aanschaffen van materialen zijn bedrijven zoals Conrad, Farnell, Texas Instruments en Mouser geschikt, omdat zij beschikken over een aanbod van hardware voor de motordriver.\\
% Voor het vragen van advies kan er terecht bij:
% \begin{itemize}
%   \item Jesse op den Brouw: C-programmeur
%   \item Stephen O'Loughlin: algemeen advies
%   \item Ad van den Bergh: programmeur
%   \item Diëgo Zuidervliet: kennis voor het aansturen van motordrivers
%   \item Paul Witte: PCB-designer
% \end{itemize}
% Voor het controleren van de (tussen)producten worden assesments gegeven met de docenten die hiervoor een aantal punten geven. Bovendien wordt er elke week een vergadering georganiseerd, waarbij de begeleider betrokken is en eventueel feedback kan geven.\\
% Het ontvangen van feedback wordt meegenomen in de toekomstige activiteiten om zo tot een beter ontwerp te komen. Daarnaast hoeft er geen rekening gehouden te worden met de regelgeving, omdat het eindproduct voor eigengebruik is. Voor de veiligheid moeten componenten gekoeld worden en niet kunnen ontploffen. Het eindproduct zal de minimale eisen hebben volgens het programma van eisen, maar de mogelijkheid bestaat dat er extra functies worden toegevoegd.\\
% Alle bovengenoemde producten moeten zorgen dat er een functionele motordriver ontworpen kan worden.



To ensure the quality of the project, the following software and resources will be utilized:

\begin{itemize}
  \item \textbf{STM32CubeIDE}: This software will be used for programming the STM32 microcontroller, which is commonly used in the industry due to its programming capabilities such as configuring GPIO and clock frequencies.
  \item \textbf{Caspoc Real-time Simulation Package}: This tool will be employed to test the operation and control of the motor driver in real-time simulations.
  \item \textbf{KiCad 7.0}: KiCad, an open-source software, will be utilized for designing PCBs for the motor driver, as it provides the necessary features for PCB design.
  \item \textbf{Eurocircuits}: PCB manufacturing will be done through Eurocircuits as per the agreements made by the school.
\end{itemize}

For procurement of materials, companies such as Conrad, Farnell, Texas Instruments, and Mouser will be considered due to their hardware offerings for the motor driver.

Expert advice can be sought from the following individuals:
\begin{itemize}
  \item Jesse op den Brouw: C programmer
  \item Stephen O'Loughlin: General advice
  \item Ad van den Bergh: Programmer
  \item Diëgo Zuidervliet: Expertise in motor driver control
  \item Paul Witte: PCB designer
\end{itemize}

Assessments will be conducted by instructors to evaluate the (intermediate) products. Additionally, weekly meetings will be held involving the supervisor to provide feedback.

Feedback received will be incorporated into future activities to enhance the design. Regulatory considerations are not necessary as the end product is for personal use. However, components must be cooled for safety, and precautions must be taken to prevent explosions. The final product will meet the minimum requirements outlined in the specifications, with the possibility of additional features being added.

All the resources will contribute to the development of a functional motor driver.

